
\chapter{整数的可除性}
\section{整除的概念与带余数除法}

\subsection*{内容概要}

\begin{definition}
  \label{def:exact-division-definition}
  设$a$、$b$是任意两个整数,其中$b \neq 0$,如果存在一个整数$q$使得
  \begin{equation}
    \label{eq:exact-division-definition}
    a=bq
  \end{equation}
$a=bq$
成立,我们就说$b$整除$a$,或者$a$被$b$整除,记作$b \mid a$,此时我们把$b$叫做$a$的因数,把$a$叫做$b$的倍数。

如果\ref{eq:exact-division-definition}中的整数$q$不存在,我们就说$b$不能整除$a$,或者说$a$不能被$b$整除,记作$b \nmid a$.
\end{definition}

\begin{theorem}
  \label{theorem:transitivity-of-exact-division}
  若$a$是$b$的倍数,$b$是$c$的倍数,则$a$是$c$的倍数。也就是
  \begin{equation}
    \label{eq:transitivity-of-exact-division}
    b \mid a, c \mid b \Rightarrow c \mid a
  \end{equation}
\end{theorem}

\begin{proof}[证明]
  依整除定义,分别存在着整数$p$和$q$,使得$a=pb$以及$b=qc$成立,因而$a=(pq)c$,结论成立。
\end{proof}

\begin{theorem}
  \label{theorem:addivity-of-exact-division}
若$a$、$b$都是$m$的倍数,则$a \pm b$也是$m$的倍数。
\end{theorem}

\begin{proof}[证明]
  由$a \pm b=q_1b \pm q_2 b = (q_1 \pm q_2)b$即证. 
\end{proof}

\begin{theorem}
  \label{theorem:linearity-of-exact-division}
若$a_1,a_2,\ldots,a_n$都是$m$的倍数,$q_1,q_2,\ldots,q_n$是任意$n$个整数,则$\sum_{i=1}^nq_ia_i$也是$m$的倍数。
\end{theorem}

\begin{proof}[证明]
  略.
\end{proof}

\begin{theorem}
  \label{theorem:division-with-remainder}
若$a$、$b$是两个整数,其中$b>0$,则存在着两个整数$q$和$r$,使得
\begin{equation}
  \label{eq:division-with-remainder}
  a=qb+r, 0 \leqslant r < b
\end{equation}
成立,而且$q$及$r$是唯一的.
\end{theorem}

\begin{proof}[证明]
  将实数集合划分为一个左闭右开区间的无穷序列:
\[ \cdots,[-2b,-b),[-b,0),[0,b),[b,2b),\cdots \]
则整数$a$必定从属于某一个确定的区间$[qb,(q+1)b)$,即$qb \leqslant a <(q+1)b$,取$r=a-qb$,则$a=qb+r$,因此$q$及$r$是存在的,而且是唯一的。
\end{proof}

\begin{definition}
  \label{def:incomplete-quotient-and-remainder}
\ref{eq:division-with-remainder}中的$q$叫做$a$被$b$除所得的不完全商,$r$叫做$a$被$b$除所得到的余数。
\end{definition}

\subsection*{习题解答}
1. 证明定理\ref{theorem:linearity-of-exact-division}.

\begin{proof}[证明]
既然$a_i(i=1,2,\dots,n)$都是$m$的倍数,就存分别存在整数
$k_i(i=1,2,\dots,n)$,使得$a_i=k_im$,因此
$\sum_{i=1}^nq_ia_i=\sum_{i=1}^nq_ik_im=m\sum_{i=1}^nq_ik_i$,得证.
\end{proof}

2. 证明$3 \mid n(n+1)(2n+1)$.其中$n$是任何整数.

\begin{proof}[证明]
$n$和$n+1$是两个相邻的正整数,如果它俩中有一个是3的倍数,那么结
论是成立的,如果都不是3的倍数,则必然是$n$除3余1,$n+1$除以3余2,这时
作为它俩之和的$2n+1$就必然是3的倍数,所以结论成立.
\end{proof}

题外话:实际上更强的结论是$6 \mid n(n+1)(2n+1)$,它是自然数的平方和:
\[ \sum_{k=1}^n\frac{1}{k^2}=\frac{1}{6}n(n+1)(2n+1) \]

3. 若$ax_0+by_0$是形如$ax+by$($x$,$y$是任意整数,$a$,$b$是两个不全为零
的整数)的数中的最小正数,则
\begin{displaymath}
  (ax_0+by_0)|(ax+by)
\end{displaymath}
其中$x$,$y$是任何整数.

\begin{proof}[证明]
根据带余除法,对于任何一对整数$x$和$y$,都存在一个整数$q$和$r$,
使得$ax+by=q(ax_0+by_0)+r$,其中$0\leqslant r<ax_0+by_0$,下面说明$r$
必定为零,因为如若不然$r=a(x-qx_0)+b(y-qy_0)$就将是比$ax_0+by_0$更小的
$a$与$b$的正的线性组合,与已知条件矛盾,所以只能$r=0$.
\end{proof}

题外话:此题目所描述的场景更加有意思的是,如果令$r=ax_0+by_0$,那么$r$是$a$和
$b$的最大公约数,并且$a$跟$b$的线性组合$ax+by$按照大小顺序排列开会是一
条等差序列:$\cdots,-2r,-r,0,r,2r,\cdots$,先证明后半部分是等差序列的
部分,因为$ax_0+by_0$是$a$与$b$的线性组合中的最小正数,所以可以肯定的
是任何两个线性组合的差(大的减小的)绝不可能小于$ax_0+by_0$,并且只能是
$ax_0+by_0$的整数倍。再来证明$ax_0+by_0$是$a$与$b$的最大公约数,根据题
目所证的结论,公约数是肯定的,为什么是最大公约数呢?根据辗转相除法知道,
最大公约数是可以用两个数的线性组合表达出来的,因此$ax_0+by_0$一定是这
个最大公约数的约数,但反过来,最大公约数肯定也能整除$ax_0+by_0$,所以
它俩只能是相等的。

这个场景改用这样的描述似乎更好一些:首先,根据辗转相除法,两个数的最大
公约数可以表达成这两个数的线性组合,其次,它还是这两个数的正的线性组合
中最小的,为什么呢,因为如果还有一个正的线性组合比它更小,那它也能被这
个最大公约数整除,但小的正整数是不能被比它大的正整数整除的,所以这是不
可能的。最后,所有线性组合都是这个最大公约数的倍数,因为它们都能被它整
除,因此如果将它们从小到大排序,就会是一个双端的无穷等差序列,相邻两项
的差就是这个最大公约数。

4. 若$a$,$b$是任意二整数,且$b\neq 0$,证明:存在两个整数$s$,$t$使得
\begin{displaymath}
  a=bs+t,|t|\leqslant \frac{|b|}{2}
\end{displaymath}
成立,并且当$b$是单数时,$s$,$t$是唯一存在的,当$b$是双数时结果如何?

\begin{proof}[证明]
只要在带余数除法中对$r>\frac{|b|}{2}$的情况增减$s$的值即可,在
$b$是双数时,如果$a$除以$|b|$的余数为$\frac{1}{2}|b|$时会存在一对这样的$s$和$t$.
\end{proof}

5. 证明$1+\frac{1}{2}+\cdots + \frac{1}{n}(n>1)$及
$\frac{1}{3}+\frac{1}{5}+\cdots + \frac{1}{2n+1}(n\geqslant 1)$都不是
整数.

\begin{proof}[证明]
对正整数$i$,将它的因数中的2全部分解出来:$i=2^{\lambda_i}l_i$,
其中$l_i$是奇数,$i=1,2,\dots,n$,记集合${\lambda_i}$中最大的值为
$\lambda$,在$n>1$时有$\lambda>0$,并且这个最大值只出现一次,因为若不然
的话,就有$i=2^{\lambda}l_i$和$j=2^{\lambda}l_j$,然而$l_i$和$l_j$都是奇数,
因此它两者之间存在着一个偶数,也就是在$i$和$j$之间还存在着一个数
$k=2^{\lambda + 1}l_k$,这与$\lambda$的最大构成矛盾,所以这个最大值只
能出现一次。
现在令
\begin{displaymath}
  S_n=1+\frac{1}{2}+\cdots + \frac{1}{n}
\end{displaymath}
两端同乘以$2^{\lambda-1}l_1l_2\cdots l_n$,得
\begin{displaymath}
  2^{\lambda-1}l_1l_2\cdots l_nS_n=Q+\frac{P}{2}
\end{displaymath}
其中后一项是使得$\lambda_i$达到最大的那一项,其余各项归于$Q$,$Q$自然
是一个整数,$P=l_1l_2\cdots l_n/l_m$自然也是整数,而且还是个奇数,因此,
等式的成立就能证明$S_n$不能是整数。
\end{proof}

\section{最大公因数与辗转相除法}

定理5 设$a$,$b$是任意两个不全为零的整数,则:
(i) 若$m$是任意一个正整数,有
\begin{displaymath}
  (am,bm)=m(a,b)
\end{displaymath}
(ii) 若$\delta $是$a$,$b$的任一公因数,有
\begin{displaymath}
  (\frac{a}{\delta},\frac{b}{\delta})=\frac{(a,b)}{\delta}
\end{displaymath}
因而
\begin{displaymath}
  (\frac{a}{(a,b)},\frac{b}{(a,b)})=1
\end{displaymath}

1. 证明推论4.1: 整数$a$与$b$的公因数与$(a,b)$的因数相同.

2. 应用上一节习题3证明$(a,b)=ax_0+by_0$,其中$ax_0+by_0$是形如
$ax+by$($x,y$是任意整数)的整数中的最小正数,并将此结果推广到多个整数的
情形。

3. 应用上一节习题4证明任意两整数的最大公因数存在,并说明其求法。并利用
此求法和辗转相除法实际算出(76501,9719)。

4. 证明辗转相除法中的除式个数即本节(1)式中的$n\leqslant \frac{2\log{b}}{\log{2}}$.


\section{整除的进一步性质及最小公倍数}

1. 证明两整数$a,b$互质的充分与必要条件是:存在两个整数$s,t$满足条件
$as+bt=1$。

2. 证明定理3。

3. 设$a_nx^n+a_{n-1}x^{n-1}+\cdots+a_1x+a_0$是一个整数系数多项式且
$a_0,a_n$都不是零,则它的有理根只能是以$a_0$的因数作分子以$a_n$的因数
作分母的既约分数,并由此推出$\sqrt{2}$不是有理数。